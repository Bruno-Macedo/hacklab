\hypertarget{offensive-security-oscp-exam-report}{%
\section{Offensive Security OSCP Exam
Report}\label{offensive-security-oscp-exam-report}}

\hypertarget{introduction}{%
\subsection{Introduction}\label{introduction}}

The Offensive Security Exam penetration test report contains all efforts
that were conducted in order to pass the Offensive Security exam. This
report will be graded from a standpoint of correctness and fullness to
all aspects of the exam. The purpose of this report is to ensure that
the student has a full understanding of penetration testing
methodologies as well as the technical knowledge to pass the
qualifications for the Offensive Security Certified Professional.

\hypertarget{objective}{%
\subsection{Objective}\label{objective}}

The objective of this assessment is to perform an internal penetration
test against the Offensive Security Exam network. The student is tasked
with following methodical approach in obtaining access to the objective
goals. This test should simulate an actual penetration test and how you
would start from beginning to end, including the overall report. An
example page has already been created for you at the latter portions of
this document that should give you ample information on what is expected
to pass this course. Use the sample report as a guideline to get you
through the reporting.

\hypertarget{requirements}{%
\subsection{Requirements}\label{requirements}}

The student will be required to fill out this penetration testing report
fully and to include the following sections:

\begin{itemize}
\tightlist
\item
  Overall High-Level Summary and Recommendations (non-technical)
\item
  Methodology walkthrough and detailed outline of steps taken
\item
  Each finding with included screenshots, walkthrough, sample code, and
  proof.txt if applicable
\item
  Any additional items that were not included
\end{itemize}

\hypertarget{high-level-summary}{%
\section{High-Level Summary}\label{high-level-summary}}

I was tasked with performing an internal penetration test towards
Offensive Security Exam. An internal penetration test is a dedicated
attack against internally connected systems. The focus of this test is
to perform attacks, similar to those of a hacker and attempt to
infiltrate Offensive Security's internal exam systems - the THINC.local
domain. My overall objective was to evaluate the network, identify
systems, and exploit flaws while reporting the findings back to
Offensive Security.

When performing the internal penetration test, there were several
alarming vulnerabilities that were identified on Offensive Security's
network. When performing the attacks, I was able to gain access to
multiple machines, primarily due to outdated patches and poor security
configurations. During the testing, I had administrative level access to
multiple systems. All systems were successfully exploited and access
granted. These systems as well as a brief description on how access was
obtained are listed below:

\begin{itemize}
\tightlist
\item
  192.168.xx.xx (hostname) - Name of initial exploit
\item
  192.168.xx.xx (hostname) - Name of initial exploit
\item
  192.168.xx.xx (hostname) - Name of initial exploit
\item
  192.168.xx.xx (hostname) - Name of initial exploit
\item
  192.168.xx.xx (hostname) - BOF
\end{itemize}

\hypertarget{recommendations}{%
\subsection{Recommendations}\label{recommendations}}

I recommend patching the vulnerabilities identified during the testing
to ensure that an attacker cannot exploit these systems in the future.
One thing to remember is that these systems require frequent patching
and once patched, should remain on a regular patch program to protect
additional vulnerabilities that are discovered at a later date.

\hypertarget{methodologies}{%
\section{Methodologies}\label{methodologies}}

I utilized a widely adopted approach to performing penetration testing
that is effective in testing how well the Offensive Security Exam
environments is secured. Below is a breakout of how I was able to
identify and exploit the variety of systems and includes all individual
vulnerabilities found.

\hypertarget{information-gathering}{%
\subsection{Information Gathering}\label{information-gathering}}

The information gathering portion of a penetration test focuses on
identifying the scope of the penetration test. During this penetration
test, I was tasked with exploiting the exam network. The specific IP
addresses were:

\textbf{Exam Network}

\begin{itemize}
\tightlist
\item
  192.168.
\item
  192.168.
\item
  192.168.
\item
  192.168.
\item
  192.168.
\end{itemize}

\hypertarget{penetration}{%
\subsection{Penetration}\label{penetration}}

The penetration testing portions of the assessment focus heavily on
gaining access to a variety of systems. During this penetration test, I
was able to successfully gain access to \textbf{X} out of the \textbf{X}
systems.

\hypertarget{system-ip-192.168.x.x}{%
\subsubsection{System IP: 192.168.x.x}\label{system-ip-192.168.x.x}}

\hypertarget{service-enumeration}{%
\paragraph{Service Enumeration}\label{service-enumeration}}

The service enumeration portion of a penetration test focuses on
gathering information about what services are alive on a system or
systems. This is valuable for an attacker as it provides detailed
information on potential attack vectors into a system. Understanding
what applications are running on the system gives an attacker needed
information before performing the actual penetration test. In some
cases, some ports may not be listed.

\begin{longtable}[]{@{}ll@{}}
\toprule
Server IP Address & Ports Open \\
\midrule
\endhead
192.168.x.x & \vtop{\hbox{\strut \textbf{TCP}:
1433,3389}\hbox{\strut \textbf{UDP}: 1434,161}} \\
\bottomrule
\end{longtable}

\textbf{Nmap Scan Results:}

\emph{Initial Shell Vulnerability Exploited}

\emph{Additional info about where the initial shell was acquired from}

\textbf{Vulnerability Explanation:}

\textbf{Vulnerability Fix:}

\textbf{Severity:}

\textbf{Proof of Concept Code Here:}

\textbf{Local.txt Proof Screenshot}

\textbf{Local.txt Contents}

\hypertarget{privilege-escalation}{%
\paragraph{Privilege Escalation}\label{privilege-escalation}}

\emph{Additional Priv Esc info}

\textbf{Vulnerability Exploited:}

\textbf{Vulnerability Explanation:}

\textbf{Vulnerability Fix:}

\textbf{Severity:}

\textbf{Exploit Code:}

\textbf{Proof Screenshot Here:}

\textbf{Proof.txt Contents:}

\hypertarget{system-ip-192.168.x.x-1}{%
\subsubsection{System IP: 192.168.x.x}\label{system-ip-192.168.x.x-1}}

\hypertarget{service-enumeration-1}{%
\paragraph{Service Enumeration}\label{service-enumeration-1}}

\begin{longtable}[]{@{}ll@{}}
\toprule
Server IP Address & Ports Open \\
\midrule
\endhead
192.168.x.x & \vtop{\hbox{\strut \textbf{TCP}:
1433,3389}\hbox{\strut \textbf{UDP}: 1434,161}} \\
\bottomrule
\end{longtable}

\textbf{Nmap Scan Results:}

\emph{Initial Shell Vulnerability Exploited}

\emph{Additional info about where the initial shell was acquired from}

\textbf{Vulnerability Explanation:}

\textbf{Vulnerability Fix:}

\textbf{Severity:}

\textbf{Proof of Concept Code Here:}

\textbf{Local.txt Proof Screenshot}

\textbf{Local.txt Contents}

\hypertarget{privilege-escalation-1}{%
\paragraph{Privilege Escalation}\label{privilege-escalation-1}}

\emph{Additional Priv Esc info}

\textbf{Vulnerability Exploited:}

\textbf{Vulnerability Explanation:}

\textbf{Vulnerability Fix:}

\textbf{Severity:}

\textbf{Exploit Code:}

\textbf{Proof Screenshot Here:}

\textbf{Proof.txt Contents:}

\hypertarget{system-ip-192.168.x.x-2}{%
\subsubsection{System IP: 192.168.x.x}\label{system-ip-192.168.x.x-2}}

\hypertarget{service-enumeration-2}{%
\paragraph{Service Enumeration}\label{service-enumeration-2}}

\begin{longtable}[]{@{}ll@{}}
\toprule
Server IP Address & Ports Open \\
\midrule
\endhead
192.168.x.x & \vtop{\hbox{\strut \textbf{TCP}:
1433,3389}\hbox{\strut \textbf{UDP}: 1434,161}} \\
\bottomrule
\end{longtable}

\textbf{Nmap Scan Results:}

\emph{Initial Shell Vulnerability Exploited}

\emph{Additional info about where the initial shell was acquired from}

\textbf{Vulnerability Explanation:}

\textbf{Vulnerability Fix:}

\textbf{Severity:}

\textbf{Proof of Concept Code Here:}

\textbf{Local.txt Proof Screenshot}

\textbf{Local.txt Contents}

\hypertarget{privilege-escalation-2}{%
\paragraph{Privilege Escalation}\label{privilege-escalation-2}}

\emph{Additional Priv Esc info}

\textbf{Vulnerability Exploited:}

\textbf{Vulnerability Explanation:}

\textbf{Vulnerability Fix:}

\textbf{Severity:}

\textbf{Exploit Code:}

\textbf{Proof Screenshot Here:}

\textbf{Proof.txt Contents:}

\hypertarget{system-ip-192.168.x.x-3}{%
\subsubsection{System IP: 192.168.x.x}\label{system-ip-192.168.x.x-3}}

\hypertarget{service-enumeration-3}{%
\paragraph{Service Enumeration}\label{service-enumeration-3}}

\begin{longtable}[]{@{}ll@{}}
\toprule
Server IP Address & Ports Open \\
\midrule
\endhead
192.168.x.x & \vtop{\hbox{\strut \textbf{TCP}:
1433,3389}\hbox{\strut \textbf{UDP}: 1434,161}} \\
\bottomrule
\end{longtable}

\textbf{Nmap Scan Results:}

\emph{Initial Shell Vulnerability Exploited}

\emph{Additional info about where the initial shell was acquired from}

\textbf{Vulnerability Explanation:}

\textbf{Vulnerability Fix:}

\textbf{Severity:}

\textbf{Proof of Concept Code Here:}

\textbf{Local.txt Proof Screenshot}

\textbf{Local.txt Contents}

\hypertarget{privilege-escalation-3}{%
\paragraph{Privilege Escalation}\label{privilege-escalation-3}}

\emph{Additional Priv Esc info}

\textbf{Vulnerability Exploited:}

\textbf{Vulnerability Explanation:}

\textbf{Vulnerability Fix:}

\textbf{Severity:}

\textbf{Exploit Code:}

\textbf{Proof Screenshot Here:}

\textbf{Proof.txt Contents:}

\hypertarget{system-ip-192.168.x.x-4}{%
\subsubsection{System IP: 192.168.x.x}\label{system-ip-192.168.x.x-4}}

\textbf{Vulnerability Exploited: bof}

Fill out this section with BOF NOTES.

\textbf{Proof Screenshot:}

\textbf{Completed Buffer Overflow Code:}

Please see Appendix 1 for the complete Windows Buffer Overflow code.

\hypertarget{maintaining-access}{%
\subsection{Maintaining Access}\label{maintaining-access}}

Maintaining access to a system is important to us as attackers, ensuring
that we can get back into a system after it has been exploited is
invaluable. The maintaining access phase of the penetration test focuses
on ensuring that once the focused attack has occurred (i.e.~a buffer
overflow), we have administrative access over the system again. Many
exploits may only be exploitable once and we may never be able to get
back into a system after we have already performed the exploit.

\hypertarget{house-cleaning}{%
\subsection{House Cleaning}\label{house-cleaning}}

The house cleaning portions of the assessment ensures that remnants of
the penetration test are removed. Often fragments of tools or user
accounts are left on an organization's computer which can cause security
issues down the road. Ensuring that we are meticulous and no remnants of
our penetration test are left over is important.

After collecting trophies from the exam network was completed, I removed
all user accounts and passwords as well as the Meterpreter services
installed on the system. Offensive Security should not have to remove
any user accounts or services from the system.

\hypertarget{additional-items}{%
\section{Additional Items}\label{additional-items}}

\hypertarget{appendix---proof-and-local-contents}{%
\subsection{Appendix - Proof and Local
Contents:}\label{appendix---proof-and-local-contents}}

\begin{longtable}[]{@{}lll@{}}
\toprule
IP (Hostname) & Local.txt Contents & Proof.txt Contents \\
\midrule
\endhead
192.168.x.x & hash\_here & hash\_here \\
192.168.x.x & hash\_here & hash\_here \\
192.168.x.x & hash\_here & hash\_here \\
192.168.x.x & hash\_here & hash\_here \\
192.168.x.x & hash\_here & hash\_here \\
\bottomrule
\end{longtable}

\hypertarget{appendix---metasploitmeterpreter-usage}{%
\subsection{Appendix - Metasploit/Meterpreter
Usage}\label{appendix---metasploitmeterpreter-usage}}

For the exam, I used my Metasploit/Meterpreter allowance on the
following machine: \texttt{192.168.x.x}

\hypertarget{appendix---completed-buffer-overflow-code}{%
\subsection{Appendix - Completed Buffer Overflow
Code}\label{appendix---completed-buffer-overflow-code}}

\begin{verbatim}
code here
\end{verbatim}
